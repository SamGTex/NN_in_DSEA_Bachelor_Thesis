\thispagestyle{plain}

\section*{Kurzfassung}
Physiker und Physikerinnen werden ständig mit inversen Problemen konfrontiert.
%Auf Lösungen inverser Probleme ist zum Beispiel das Hochenergie-Neutrino-Observatorium \texit{IceCube} angewiesen.
Über die Zeit wurden verschiedene Methoden zur Lösung entwickelt.
Der \textbf{D}ortmund \textbf{S}pectrum \textbf{E}stimation \textbf{A}lgorithm (DSEA) fasst das zugrundeliegende Problem als Klassifikationsaufgabe auf.
Dabei können verschiedene Klassifikationsalgorithmen verwendet werden.
Der Einsatz von neuronalen Netzen als Klassifizierer in DSEA ist ein neuer Ansatz und wird auf IceCube-Neutrino Simulationen untersucht.
Neuronale Netze in DSEA bieten die Möglichkeit den Trainingsprozess weiter fortzuführen, statt einen Klassifizierer in jeder Iteration neu zuinitalisieren.
Systematisch werden die Hyperparameter des neuronalen Netzes und von DSEA untersucht.
Ein häufiges Problem neuronaler Netze ist die Überanpassung an die Trainingsdaten.
Es wird geprüft, ob die Modelle eine unabhängige Vorhersage treffen.

\section*{Abstract}
\begin{foreignlanguage}{english}
Physicists are frequently confronted with inverse problems.
Over time, various methods have been developed to solve them.
The \textbf{D}ortmund \textbf{S}pectrum \textbf{E}stimation \textbf{A}lgorithm (DSEA) treats the fundamental problem as a classification task.
Various classification algorithms can be used.
The use of neural networks as classifier in DSEA is a new approach and is explored on IceCube neutrino simulations.
Neural networks in DSEA offer the possibility to continue the training process, instead of reinitializing the classifier in each iteration.
Systematically, the hyperparameters of the neural network and DSEA are investigated.
A common problem of neural networks is overfitting to the training data.
The models are tested for unbiased predictions.
\end{foreignlanguage}
