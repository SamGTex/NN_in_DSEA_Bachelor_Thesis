\chapter{Verwendete Methoden}
In diesem Kapitel werden die verwendeten Methoden vorgestellt.
\autoref{sec:chi2} behandelt die $\Chi^2$-Distanz als Abstandsmaß zweier diskreter Verteilungen.
In \autoref{sec:bootstrap} wird die Berechnung der Unsicherheiten mithilfe von Bootstrapping erläutert.

\section{Chi-Quadrat-Distanz} \label{sec:chi2}
Die $\chi^2$-Distanz\cite{LIU20022617} ist ein gewichtetes Abstandsmaß und dient zur Überprüfung von Konvergenz.
Der Abstand zwischen der wahren Verteilung $f$ und der rekonstruierten Verteilung $\hat{f}$ wird über die Abstände der einzelnen Bins $j$ nach
\begin{equation}
    \chi^2 = \frac{1}{2} \sum_{j=1}^{n} \frac{(\hat{f}_j - f_j)^2}{\hat{f_j} + f_j}
\end{equation}
bestimmt.
Ist die $\chi^2$-Distanz kleiner oder gleich der $\chi^2$-Distanz in der vorherigen DSEA-Iteration, kann das Verfahren abgebrochen werden.

\section{Bootstrapping: Abschätzung der Unsicherheiten} \label{sec:bootstrap}
Bootstrapping\cite{lewis_poole_2004} ist eine Methode zur Bestimmung der Unsicherheiten von Wahrscheinlichkeitsmodellen.
Sie ist sinnvoll, wenn normalerweise nur ein kleiner Teil des Trainingsdatensatzes entfaltet wird.
\\
Die Methode beruht auf dem "`Resampling"', also einer Veränderung des ursprünglichen Datensatzes durch Mehrfachzählung und Nichtbeachtung von Events.
\\
\\
In \textbf{Schritt 1} wird der Datensatz durch Ersetzung von Events variiert.
Dieser Datensatz wird in \textbf{Schritt 2} mit DSEA (siehe \autoref{sec:dsea}) entfaltet.
In \textbf{Schritt 3} wird das entfaltete Spektrum gespeichert.
Die Schritte 1-3 werden $n$-mal wiederholt.
\\
\\
Anschließend kann die Verteilung jedes einzelnen Energie-Bins untersucht werden.
Zur Bestimmung des Mittelwerts kann der Median 
\begin{equation}
    \tilde {x} =
    \begin{cases}
        x_{m+1} & \text{ ,für ungerades n = 2m+1}\\
        \frac{1}{2} (x_{m}+x_{m+1}) & \text{ ,für gerades n = 2m}
    \end{cases}
\end{equation}
für jeden Energie-Bin berechnet werden, wobei $n$ die Anzahl der Entfaltungen angibt.
Die zugehörigen Unsicherheiten können allgemein über das untere und obere Quantil angegeben werden.
Das untere Quantil beinhaltet $\SI{16}{\percent}$ und das obere Quantil $\SI{84}{\percent}$ der Datenpunkte.
Zwischen den Quantilen liegen also $~\SI{68}{\percent}$ der Messwerte.
\\
Handelt es sich um normalverteilte Daten, so sind die Quantile äquivalent zu der Standardabweichung und der Median ist gleich dem Mittelwert.