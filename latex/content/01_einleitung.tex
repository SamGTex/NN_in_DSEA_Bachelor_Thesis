\chapter{Einleitung}
Die Astroteilchenphysik untersucht aller Art Teilchen, die von extraterrestrischen Quellen ausgesandt wurden.
Ziel ist die Bestimmung der Position und der Eigenschaften von Himmelskörpern über die ankommenden Botenteilchen.
Die Botenteilchen setzen sich aus geladenen Teilchen (kosmische Strahlung), Photonen $\gamma$ und Neutrinos $\nu$ zusammen.\cite{icecube_detector}
\\
\\
IceCube ist ein Detektor am geographischen Südpol und ist dafür ausgelegt, hochenergetische Neutrinos nachzuweisen.
Zum einen ist der Ort der Neutrinoquelle von großer Interesse.
Aktive Galaxiekerne, schwarze Löcher oder auch Neutronensterne können mögliche Quellen der Neutrinos sein und sind für Physiker*innen von großem Interesse.
Eine weitere wichtige Größe stellt das Energiespektrum der Neutrinos dar.
\\
Die Neutrinoenergie ist nicht direkt messbar.
Stattdessen werden die Zerfallsprodukte der Neutrinos (Elektronen, Myonen, Tau-Leptonen) gemessen.
Das zu lösende Problem ist die Rekonstruktion der Neutrinoenergie aus den gemessenen Größen.
Es handelt sich um ein inverses Problem, welches grundsätzlich schlecht konditioniert ist.
Die Rekonstruktion der Energie wird als Entfaltung bezeichnet und benötigt spezielle Ansätze.
\\
\\
Der \textbf{D}ortmund \textbf{S}pectrum \textbf{E}stimation \textbf{A}lgorithm (DSEA) \cite{ruhe} betrachtet das Problem als Klassifikationsproblem.
Unter Verwendung von Methoden des maschinellen Lernens kann eine Entfaltung mit DSEA durchgeführt werden.
\\
In dieser Arbeit wird zum ersten mal die Verwendung von neuronalen Netzen als Klassifizierer in DSEA untersucht.
Dies bietet neue Möglichkeiten.
Die bisher verwendeten Klassifizierer wurden in jeder DSEA-Iteration neu initialisiert.
Neuronale Netze basieren auf einer Verlustfunktion.
Es ist daher möglich den Trainingsprozess eines Modells fortzuführen, statt in jeder DSEA-Iteration von neuem zu beginnen.
\\
Die Struktur eines neuronalen Netzes kann an das zugrundeliegende Problem angepasst werden.
Es gibt eine große Auswahl an Ebenen, Verlustfunktionen und Aktivierungsfunktionen.
Duch Anpassung der Hyperparameter kann die Lösung optimiert werden.